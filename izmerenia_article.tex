\providecommand{\pdfxopts}{a-1b,cyrxmp}
%\providecommand{\pdfxopts}{a-1b}
\providecommand{\thisyear}{2021}
\immediate\write18{rm \jobname.xmpdata}%  uncomment for Unix-based systems
\begin{filecontents*}{\jobname.xmpdata}
\Title{Измерение тока и напряжения в косоугольных координатах в трехфазной обобщенной электрической машине\textemdash\thisyear}
\Author{Альмушреки Осама Абду Али\sep
Обама Омбеде Николя Серж\sep
Прокшин Артем Николаевич\sep
Татаринцев Николай Иванович\sep
Трофимов Александр Викторович}
\Creator{pdfTeX + pdfx.sty with options \pdfxopts }
\Subject{Измерение тока и напряжения в косоугольных координатах в трехфазной обобщенной электрической машине}
	\Keywords{изменения фазного тока\sepизменения линейного напряжения\sepковариантные координаты\sepконтравариантные координаты}
\CoverDisplayDate{март \thisyear}
\CoverDate{2021-03-14}
\Copyrighted{True}
%\Copyright{Public Domain}
\CopyrightURL{http://github.com/trot-t}
\Creator{pdfTeX + pdfx.sty with options \pdfxopts }
\end{filecontents*}


\documentclass[a4paper,twocolumn,10pt]{article}
\usepackage[affil-it]{authblk}
%\usepackage[14pt]{extsizes}
%\usepackage[12pt]{extsizes}

\usepackage[margin=20mm]{geometry}
\geometry{
        a4paper,
        total={165mm,247mm},
        top=30mm,
        bottom=20mm
%       evenmargin=25mm,
%       oddsidemargine=20mm
%       left=20mm
}


\pdfcompresslevel=9

\usepackage[\pdfxopts]{pdfx}[2016/03/09]
\PassOptionsToPackage{obeyspaces}{url}
\let\tldocrussian=1  % for live4ht.cfg

%
% https://tavda.net/blog/latex/
% 

\usepackage[TS1,T2A]{fontenc}
\usepackage[utf8]{inputenc}
\usepackage[english,russian]{babel}
\usepackage{tikz}                                            % для чертежей
\usepackage[european,cuteinductors,smartlabels]{circuitikz}  % для электронных схем
\usetikzlibrary{calc}

\usepackage{amsmath,amsfonts}
\usepackage{amssymb}
%\usepackage[scr]{rsfso} % буквы для алгебры множеств

\usepackage{gnuplottex}   % автоматическая вставка графиков из программ моделирования электрических схем


\usepackage{enumitem}

%%% Межстрочный интервал
\usepackage{setspace}

%% таблицы в стиле старых книг
\usepackage{booktabs} 

%% для подкладывания отдельных pdf-страниц 
\usepackage{pdfpages}

%% для кода
\usepackage{listings}

\definecolor{lightgrey}{rgb}{0.9,0.9,0.9}
\definecolor{lightblue}{rgb}{0,0,1}

\definecolor{grey}{rgb}{0.5,0.5,0.5}
\definecolor{blue}{rgb}{0,0,1}
\definecolor{violet}{rgb}{0.5,0,0.5}

\definecolor{darkred}{rgb}{0.5,0,0}
\definecolor{darkblue}{rgb}{0,0,0.5}
\definecolor{darkgreen}{rgb}{0,0.5,0}


\renewcommand{\lstlistingname}{\normalsize Листинг}
\newcommand{\listfile}[1]{\lstinputlisting{#1}}
\lstdefinelanguage{ngspice}{%
    morekeywords={control, end, endc, subckt, ends, plot, gnuplot, model, temp, ac, dc, tran, options, set},%
    morekeywords={CONTROL, END, ENDC, SUBCKT, ENDS, PLOT, GNUPLOT, MODEL, TEMP, AC, DC, TRAN, OPTIONS, SET},%
    directives={},
    comment=[l]{*},
    morecomment=[l]{;},
    morestring=[b]',
    morestring=[b]"
}[directives]
\lstset{%
  language=ngspice,%
  basicstyle=\ttfamily\scriptsize,%
  sensitive=true,%
  keywordstyle=\color{blue},%
  stringstyle=\color{darkgreen},%
  commentstyle=\color{violet},%
  directivestyle=\color{blue},
  showstringspaces=false,%
  tabsize=2,%
  frame=leftline,
  rulecolor=\color{lightblue},
  numberstyle=\tiny,
  numbers=left,
  numbersep=10pt,
  xleftmargin=20pt,
  framexleftmargin=2.5mm,
  framexleftmargin=5pt,
  framesep=15pt,
  fillcolor=\color{lightgrey},
  inputencoding=cp1251, % принимает только WINDOWS-1251 кодировку
   extendedchars=true
}

%\def\No{\textnumero} % № для unix
\title{\fontsize{14}{7.2}\selectfont Измерение тока и напряжения в косоугольных координатах в трехфазной обобщенной электрической машине}
\author[1]{\small О.А. Али Альмушреки}
\author[2]{\small Н.С. Обама Омбеде}
\author[3]{\small А.Н. Прокшин}
\author[4]{\small Н.И. Татаринцев}
\author[5]{\small А.В. Трофимов}
\affil[ ]{Санкт-Петербургский государственный электротехнический университет «ЛЭТИ» им. В.И. Ульянова (Ленина)}
\affil[ ]{\textit {\textsuperscript{1}eng.os.mu@gmail.com, \textsuperscript{2}sergeobama8421@gmail.com,\textsuperscript{3}anprokshin@etu.ru, \textsuperscript{4}ntatarintsev@yandex.ru, \textsuperscript{5}a7trofimov@gmail.com}}
% eng.os.mu@gmail.com, 2sergeobama8421@gmail.com,3anprokshin@etu.ru, 4ntatarintsev@yandex.ru, 5a7trofimov@gmail.com
\date{} % clear date
% Конец преамбулы

\begin{document}
\pagenumbering{gobble} % не нумеруем страницы
\maketitle
% шаблон графика параболы
%\begin{tikzpicture}
%\newcommand{\xb}{-3}
%\newcommand{\xa}{3}
%\draw[thin, ->] (-6,0) -- (6,0) node[right] {$X$};
%\draw[thin, ->] (0,-6) -- (0,6) node[left] {$Y$};
%\foreach \x\xtext in {-5/-5,5/5,{\xb}/\xb,{\xa}/{\displaystyle \frac{-b+\sqrt{b^2-4ac}}{2a}}} % 
%   \draw (\x,0.1) -- (\x,-0.1) node[below] {$\xtext$};
%\draw[domain=-5:5, help lines, smooth]
%       plot ({\x},{0.2*(\x-\xa)*(\x-\xb)});
%\end{tikzpicture}
%\vspace{-1cm}
\begin{abstract}
{\bf\small
Измерения линейных напряжений и фазных
токов в трехфазной электрической машине без нулевого
провода приведенные к одной комплексной плоскости
оказываются ковариантными и контравариантными
координатами в косоугольной системе координат. При этом
оси линейных напряжений и фазных токов оказываются
взаимосопряженными.

Физические величины активной и реактивной мощности
оказываются пропорциональны скалярному и векторному
произведению векторов, в которых координаты векторов в
произведениях должны браться из взаимосопряженных осей.

В настоящее время общепринятый метод лежащий в
основе векторного управления электрической машиной
включает в себя последовательность преобразований
Кларка, Парка-Горева, обратного преобразования Парка-Горева, 
	обратного преобразования Кларка, что является, по
существу, переходом к декартовой системе координат и
затем обратно к косоугольной.

Предъявлен метод построения осей: ось ''d-'' -- это
линейная комбинация ковариантных координат выбранного
	вектора ${\bf (X1, X2)}$, а ось ''q-'' линейная комбинация
	ковариантных координат ${\bf (X2, -X1)}$ или контравариантных
координат. Благодаря этому удается построить систему
управления электрической машины без перехода в
декартову систему и обратно, что позволяет существенно
	сократить объем вычислений.}
\end{abstract}
\\

{\bf\it\small Ключевые слова: измерение фазных токов, измерение линейных напряжений, ковариантные координаты, контравариантные координаты}
\\

Рассматривается метод измерения линейных напряжений и токов. Рассмотрение производится применительно к симметричной трехфазной системе.
В математическом описании используются косоугольные системы координат и понятия ковариантности и контравариантности с общепринятым порядком расположения индексов.
Индексы сверху относятся к контравариантным компонентам, а индексы снизу к  ковариантным. 

Проведем измерения мгновенных значений линейных напряжений $u_{\scriptscriptstyle AB}$,   $u_{\scriptscriptstyle BC}$ и фазных токов $(-i_{\scriptscriptstyle A})$, 
$i_{\scriptscriptstyle C}$ в активном выпрямителе, подключенном к трехфазной электрической сети без нулевого провода. Подключение активного выпрямителя к трехфазной сети и измерительных
приборов приведено на~рис.~\ref{inverter_grid_IV}.

%\subsection*{Измерение тока и напряжения в косоугольных координатах в трехфазной обобщенной электрической машине}
\begin{figure}[!ht]
\centering
\begin{circuitikz}[scale=1]
\ctikzset{bipoles/length=1.0cm}
%\ctikzset{iloop /.style={width=2.0}}
\draw
(1.25,2.65)node[nigbt,bodydiode](npn1){}% 1 ряд 
%\node[nigbt,bodydiode] (npn1) at (1.25,3.1) {};% 1 ряд 
(1.25,.55) node[nigbt,bodydiode](npn4){}%1ряд
(npn1.S) -- (npn4.D);

% найдем положение плюсовой шины
\path let \p1 = (npn1.D) in node(plus)  at (0,\y1) {};
\draw (0,0) to[C] (plus)

(2.75,2.65)node[nigbt,bodydiode](npn3){}% 2 ряд 
(2.75,.55) node[nigbt,bodydiode](npn6){}% 2ряд
(npn3.S) -- (npn6.D)

(4.25,2.65)node[nigbt,bodydiode](npn5){}%последний ряд 
(4.25,.55) node[nigbt,bodydiode](npn2){}%последний ряд
(npn5.S) -- (npn2.D)

(plus.center) --(npn1.D) node[above]{} -- (npn3.D) node[above]{} -- (npn5.D) node[above]{} % плюсовая шина
(0,0) -- (npn4.S) node[below]{} -- (npn6.S) node[below]{} -- (npn2.S) node[below]{} % минусовая шина

($(npn5.S)!0.85!(npn2.D)$)   node[left]{\scriptsize$C$} to[short,-] ++ (0.5,0) to[iloop, name=Ic, mirror] ++(0.5,0) to[L,american inductor,-] ++ (1.5,0) --++ (1.5,0)  node(VC) {} 
	 --++ (.5,0) node(C) {}
	($(npn3.S)!0.5!(npn6.D)$) node[left]{\scriptsize$B$} to[short,-] ($(npn5.S)!0.5!(npn2.D)$) --++ (1.0,0) to[L,american inductor,-] ++ (1.5,0) --++(.5,0) node(VB) {} --++ (.5,0) node(VB1){} 
	--++ (1,0) % катуха В 
	($(npn1.S)!0.15!(npn4.D)$) node[left]{\scriptsize$A$}  to[iloop, name=Ia, mirror,-] ++(1.5,0) to[short] ($(npn5.S)!0.15!(npn2.D)$) --++ (1,0) to[L,american inductor,-] 
	++ (1.5,0) node(VA) {} --++ (2,0) node(A) {}

(A.center)--(C.center) (A.center) --++(0,1.5) (C.center) --++ (0,-1.5)

(A.center) ++(0.1,0.8) --++ (-0.2,0.2)
(A.center) ++(0.1,0.9) --++ (-0.2,0.2)
(A.center) ++(0.1,1.0) --++ (-0.2,0.2)

(Ia) --++ (0,-2.2) node[below] {\small $i_{\scriptscriptstyle A}$}
(Ic) --++ (0,-1.7) node[below] {\small $i_{\scriptscriptstyle C}$}

%(VB.center) --++(0,-1) to[smeter,t=v,l_=$\small v_{\scriptscriptstyle BC}$] ++(1,0) -- (VC.center)
(VB.center) --++(0,-1) to[rmeterwa,t=v,l_=$\small u_{\scriptscriptstyle BC}$] ++(1,0) -- (VC.center)
(VA.center)  --++(0,1) to[rmeterwa,t=v,l^=$\small u_{\scriptscriptstyle AB}$] ++(1,0) -- (VB1.center);
%(VA.center)  --++(0,1) to[smeter,t=v,l_=$\small v_{\scriptscriptstyle AB}$] ++(1,0) -- (VB1.center);
        \draw[<-] (4.9,3.9) -- (6.5,3.9);
	\draw[->] (4.9,4.1) -- (6.5,4.1) node[midway, above] {$p,q$};
;\end{circuitikz}

\caption{схема измерения токов и напряжений в трехфазном активном выпрямителе}
\label{inverter_grid_IV}
\end{figure}

%Рассмотрим результаты измерений мгновенных значений линейных напряжений в косоугольной системе кооринат с углом между осями  $v_{\scriptscriptstyle AB}$  и $v_{\scriptscriptstyle BC}$ равным 120 электрических градусов.
%Изображающий вектор линейного напряжения
Измерения мгновенных значений линейных напряжений есть перпендикулярные проекции изображающего вектора $\vec{u}$ на оси $U_{\scriptscriptstyle AB}$  и $U_{\scriptscriptstyle BC}$ угол 
между которыми 120 градусов.

Рассмотрим косоугольную систему координат  $U_{\scriptscriptstyle AB}$  и $U_{\scriptscriptstyle BC}$ с единичными векторами  $\vec{e}_{\scriptscriptstyle\!AB}$  и $\vec{e}_{\scriptscriptstyle\!BC}$.
Перпендикулярные координаты обозначим $u_{\scriptscriptstyle\!AB}$ и $u_{\scriptscriptstyle\!BC}$.
Разложение вектора $\vec{u}$ по векторам  $\vec{e}_{\scriptscriptstyle\!AB}$  и $\vec{e}_{\scriptscriptstyle\!BC}$ по правилу параллелограмма:

\begin{equation}
\vec{v} =  u^{\scriptscriptstyle\!AB} \vec{e}_{\scriptscriptstyle\!AB} + u^{\scriptscriptstyle\!BC} \vec{e}_{\scriptscriptstyle\!BC}
\label{vector_eq}
\end{equation}

%где  $v^{\scriptscriptstyle\!AB}$, $v^{\scriptscriptstyle\!BC}$ (индекс сверху) координаты разложения.

\begin{figure}[!ht]
\centering
	\begin{circuitikz}[scale=0.8]
	\newcommand{\Axis}{6.3}
	\newcommand{\Axisy}{3.8}
	\newcommand{\Axisyy}{-0.9}
	\newcommand{\gammaa}{120} % угол между осями
	\newcommand{\E}{2.7}
	\newcommand{\alfa}{20} % угол вектора
	\newcommand{\V}{4.5}
	\draw[thin,->] (0,0) -- ({\Axis*cos(0)},{\Axis*sin(0)}) node [right] {$U_{\scriptscriptstyle AB}$};
	\draw[thin,->] ({\Axisyy*cos(\gammaa)},{\Axisyy*sin(\gammaa)}) -- ({\Axisy*cos(\gammaa)},{\Axisy*sin(\gammaa)}) node [left] {$U_{\scriptscriptstyle BC}$};
	\draw[thick, ->] (0,0) -- ({\E*cos(0)},{\E*sin(0)}) node[below] {$\vec{e}_{\scriptscriptstyle AB}$};
	\draw[thick, ->] (0,0) -- ({\E*cos(\gammaa)},{\E*sin(\gammaa)}) node[left] {$\vec{e}_{\scriptscriptstyle BC}$};
        % сам вектор
	\draw[blue,->] (0,0) -- ({\V*cos(\alfa)},{\V*sin(\alfa)}) node (V) {} node[right] {$\vec{u}$};
	% перпендикулярные проекции
	\draw[dashed] ({\V*cos(\alfa)},{\V*sin(\alfa)}) -- ({\V*cos(\alfa)},0) node[below] {$u_{\scriptscriptstyle\!A\!B}$};
	\newcommand{\Vbc}{(\V*cos(\alfa)*cos(\gammaa) + \V*sin(\alfa)*sin(\gammaa))} % проекция на ось Vbc
%	\draw[dashed] (V.center) -- ({\Vbc*cos(\gammaa)},{\Vbc*sin(\gammaa)}) node[below left=-1mm] {$v_{\scriptscriptstyle\!B\!C}$};
	\draw[dashed] (V.center) -- ({\Vbc*cos(\gammaa)},{\Vbc*sin(\gammaa)}) node[left=1mm] {$u_{\scriptscriptstyle\!B\!C}$};

	%проекция на сопряженную ось e^{AB}
	\newcommand{\VAB}{(\V*cos(\alfa)*cos(\gammaa-90-\alfa) + \V*sin(\alfa)*sin(\gammaa-90-\alfa))} % проекция на ось V^AB
	\draw[dashed]  (V.center) -- ({\VAB/cos(\gammaa-90)}, 0) node[below] {$u^{\scriptscriptstyle\!A\!B}$};
	%проекция на сопряженную ось e^{BC}
	\newcommand{\VBC}{\V*sin(\alfa)}
	\draw[dashed]  (V.center) --  ({\VBC/cos(\gammaa-90)*cos(\gammaa)}, {\VBC/cos(\gammaa-90)*sin(\gammaa)}) node[below left=-1mm] {$u^{\scriptscriptstyle\!B\!C}$};
\end{circuitikz}
	\caption{ковариантные и контравариантные координаты изображающего вектора напряжения}
\end{figure}

Квадрат длины вектора $\vec{u}$ равен:
$$
|\vec{u}|^2 = u_{\scriptscriptstyle AB}u^{\scriptscriptstyle AB} +  u_{\scriptscriptstyle BC}u^{\scriptscriptstyle BC}
$$

Квадрат длины вектора с другой стороны равен скалярному произведению:
\begin{equation}
(\vec{u} \cdot \vec{u})  = u_{\scriptscriptstyle AB}u^{\scriptscriptstyle AB} +  u_{\scriptscriptstyle BC}u^{\scriptscriptstyle BC}
\label{our_basic}
\end{equation}


Здесь один из множителей является измеренным мгновенным значением линейного напряжения с нижним индексом, другой -- математическим разложением изображающего вектора с верхним индексом.
%, или формально в произведении у одной координаты индекс нижний, у другой -- верхний.


Просуммировав и усреднив формулы~(\ref{vector_eq}) во всех парах осей $U_{\scriptstyle AB}, U_{\scriptstyle BC}$ и $U_{\scriptstyle BC}, U_{\scriptstyle CA}$ и 
$U_{\scriptstyle CA}, U_{\scriptstyle AB}$, и заметив, что полусумма контравариантных (неизмеряемых) координат $u^{\scriptscriptstyle AB} $ в осях $U_{\scriptstyle AB}, U_{\scriptstyle BC}$  
и %контравариантной координаты 
$u^{\scriptscriptstyle AB} $ в осях $U_{\scriptstyle CA}, U_{\scriptstyle AB}$ в случае симметричной системы равна ковариантной (измеряемой) координате $u_{\scriptscriptstyle AB}$ \cite{Prokshin}, получаем
известную \cite{Gorev},\cite{Sokolovsky} формулу:
\begin{equation}
\vec{u} = \frac{2}{3} \left(
u_{\scriptscriptstyle\!AB} \vec{e}_{\scriptscriptstyle\!AB} + u_{\scriptscriptstyle\!BC} \vec{e}_{\scriptscriptstyle\!BC} + u_{\scriptscriptstyle\!CA} \vec{e}_{\scriptscriptstyle\!CA}
\right)
\label{park-gorev}
\end{equation}

%здесь все индексы нижние. Однако формула (\ref{vector_eq}) верна и для несимметричных систем~\cite{Prokshin}.

В геометрии к косоугольному базису единичных векторов $\vec{e}_{\scriptscriptstyle 1}, \vec{e}_{\scriptscriptstyle 2}$ можно построить двойственный базис 
$\vec{e}^{\scriptscriptstyle\;1}, \vec{e}^{\scriptscriptstyle\;2}$ \cite{Borisenko}
по правилу:
$$
\begin{array}{ll}
	(\vec{e}_{\scriptscriptstyle 1}\cdot \vec{e}^{\scriptscriptstyle\;1}) = 1; &  (\vec{e}_{\scriptscriptstyle 1}\cdot \vec{e}^{\scriptscriptstyle\;2}) = 0;\\
	(\vec{e}_{\scriptscriptstyle 2}\cdot \vec{e}^{\scriptscriptstyle\;1}) = 0; &  (\vec{e}_{\scriptscriptstyle 2}\cdot \vec{e}^{\scriptscriptstyle\;2}) = 1
\end{array}
$$
\begin{figure}[!ht]
\centering
	\begin{circuitikz}[scale=0.85]
        \newcommand{\Axis}{6.3}
        \newcommand{\Axisy}{4.5}
        \newcommand{\Axisyy}{-1.4}
        \newcommand{\gammaa}{120} % угол между осями
        \newcommand{\E}{3.2}
        \newcommand{\alfa}{20} % угол вектора
        \newcommand{\V}{4.5}
        \draw[thin,->] (0,0) -- ({\Axis*cos(0)},{\Axis*sin(0)}) node [right] {$\small X_{\scriptstyle 1}$};
        \draw[thin,->] ({\Axisyy*cos(\gammaa)},{\Axisyy*sin(\gammaa)}) -- ({\Axisy*cos(\gammaa)},{\Axisy*sin(\gammaa)}) node [left] {$\small X_{\scriptstyle 2}$};
        \draw[thick, ->] (0,0) -- ({\E*cos(0)},{\E*sin(0)}) node[below] {$\vec{e}_{\scriptstyle 1}$};
        \draw[thick, ->] (0,0) -- ({\E*cos(\gammaa)},{\E*sin(\gammaa)}) node[left] {$\vec{e}_{\scriptstyle 2}$};
	% сопряженные оси
	\draw[thin,->] (0,0) -- ({\Axis*cos(\gammaa-90)},{\Axis*sin(\gammaa-90)}) node [right] {$\small X^{\scriptstyle 1}$};
	\draw[thin,->] ({\Axisyy*cos(90)},{\Axisyy*sin(90)}) -- ({\Axisy*cos(90)},{\Axisy*sin(90)}) node [left] {$\small X^{\scriptstyle 2}$};
	\draw[thick, ->] (0,0) -- ({\E*cos(0)},{\E*tan(\gammaa-90)}) node[above left=-1.5mm] {$\vec{e}^{\scriptstyle\;1}$};
	\draw[thick, ->] (0,0) -- (0, {\E/cos(\gammaa-90)}) node[right] {$\vec{e}^{\scriptstyle\;2}$};

       % сам вектор
        \draw[thick, blue] (0,0) -- ({\V*cos(\alfa)},{\V*sin(\alfa)}) node (V) {} node[right] {$\vec{x}$};
        % перпендикулярные проекции
	\draw[dashed] ({\V*cos(\alfa)},{\V*sin(\alfa)}) to[short,-] ({\V*cos(\alfa)},0) node[below] {$\frac{x_{\scriptscriptstyle 1}}{|e_{\scriptscriptstyle 1}|}$};
	\filldraw[color=white, draw=black] ({\V*cos(\alfa)},0)  circle (0.56mm);
	% продолжим до сопряженной оси
	\draw[dashed] ({\V*cos(\alfa)},{\V*sin(\alfa)}) to[short] ({\V*cos(\alfa)},{\V*cos(\alfa)*tan(\gammaa-90)}) node[right] {${\scriptstyle x_{\scriptscriptstyle 1}|e^{\scriptscriptstyle 1}|}$};
	\filldraw[color=white, draw=black] ({\V*cos(\alfa)},{\V*cos(\alfa)*tan(\gammaa-90)}) circle (0.56mm);

        \newcommand{\Vbc}{(\V*cos(\alfa)*cos(\gammaa) + \V*sin(\alfa)*sin(\gammaa))} % проекция на ось Vbc
	% продолжим до вертикальной оси
	\draw[dashed] (V.center) -- (0,{\Vbc/cos(\gammaa-90)}) node[left] {${\scriptstyle x_{\scriptscriptstyle 2} |e^{\scriptscriptstyle 2}|}$};
	\filldraw[color=white, draw=black]  (0,{\Vbc/cos(\gammaa-90)})   circle (0.56mm);
	\draw[dashed] (V.center) -- ({\Vbc*cos(\gammaa)},{\Vbc*sin(\gammaa)}) node[right=1.8mm] {$\frac{x^{\scriptscriptstyle 2}}{|e^{\scriptscriptstyle 2}|}$};
	\filldraw[color=white, draw=black]  ({\Vbc*cos(\gammaa)},{\Vbc*sin(\gammaa)}) circle (0.56mm);

        %проекция на сопряженную ось e^{AB}
%        \newcommand{\VAB}{(\V*cos(\alfa)*cos(\gammaa-90-\alfa) + \V*sin(\alfa)*sin(\gammaa-90-\alfa))} % проекция на ось V^AB  -- clamsy error
        \newcommand{\VAB}{\V*cos(\gammaa-90-\alfa)} % проекция на ось V^AB
	\draw[dashed]  (V.center) -- ({\VAB/cos(\gammaa-90)}, 0) node[below] {${\scriptstyle x^{\scriptscriptstyle 1} |e_{\scriptscriptstyle 1}|}$};
        \filldraw[color=white, draw=black] ({\VAB/cos(\gammaa-90)}, 0)  circle (0.56mm);
	\draw[dashed]  (V.center) -- ({\VAB*cos(\gammaa-90)}, {\VAB*sin(\gammaa-90)}) node[above left=-1.4mm] {$\frac{x^{\scriptscriptstyle 1}}{|e^{\scriptscriptstyle 1}|} $}; 
        \filldraw[color=white, draw=black]  ({\VAB*cos(\gammaa-90)}, {\VAB*sin(\gammaa-90)})  circle (0.56mm);
	% проекции вдоль оси X_1
	\draw[dashed]  (V.center) -- ({-\V*sin(\alfa)*tan(\gammaa-90)} , {\V*sin(\alfa)}) node[left] {${\scriptstyle x^{\scriptscriptstyle 2} |e_{\scriptscriptstyle 2}|}$};
	\filldraw[color=white, draw=black]  ({-\V*sin(\alfa)*tan(\gammaa-90)} , {\V*sin(\alfa)}) circle (0.56mm);
	\draw[dashed]  (V.center) -- (0, {\V*sin(\alfa)}) node[above right] {$\frac{x^{\scriptscriptstyle 2}}{|e^{\scriptscriptstyle 2}|} $};
	\filldraw[color=white, draw=black]  (0, {\V*sin(\alfa)}) circle (0.56mm);

\end{circuitikz}
	\caption{координаты вектора в двойственных (взаимосопряженных) базисах}
	\label{coordinates}
\end{figure}

Двойственные (сопряженные) оси выбираются так, чтобы угол между исходной и сопряженной осью был острым. 
%Аналогично разложению по основному базису (\ref{vector_eq}) 
Вектор $\vec{x}$ можно разложить по двойственному (сопряженному) базису 
$$
\vec{x} =  x_1 \vec{e}^{\;1} + x_2 \vec{e}^{\;2}
$$
Координаты вектора $\vec{x}$ при произвольной нормировке базовых векторов приведены на рис. \ref{coordinates}.


Скалярное произведение двух произвольных векторов равно
\begin{equation}
\left(\vec{x}\cdot\vec{y}\right) = x_1y^1 +  x_2y^2  = x^1y_1 +  x^2y_2
\label{scalar}
\end{equation}
В нашей физической системе выбирем оси линейных напряжений и фазных токов как указано на рис.\ref{pickup_mesure}. 
Оси %, на которых откладываются измерения 
линейных напряжений $U_{\scriptscriptstyle AB}, U_{\scriptscriptstyle BC}$ и 
фазных токов $I_{\scriptscriptstyle (-A)}, I_{\scriptscriptstyle C}$
оказываются взаимосопряженными в геометрическом смысле, за исключением того что оси тока и напряжения строятся на одном графике и нормировка базовых векторов
такова: $(\vec{e}_{\scriptscriptstyle u\,AB}\cdot \vec{e}_{\scriptscriptstyle u\,AB})=1$,
$(\vec{e}_{\scriptscriptstyle u\,BC}\cdot \vec{e}_{\scriptscriptstyle u\,BC})=1$,  
$(\vec{e}_{\scriptscriptstyle i\,(-A)}\cdot \vec{e}_{\scriptscriptstyle i\,(-A)})=1$ и
$(\vec{e}_{\scriptscriptstyle i\,C}\cdot \vec{e}_{\scriptscriptstyle i\,C})=1$. 
Выбор оси фазного тока $I_{(-A)}$ обусловлен тем, чтобы угол между взаимосопряженными осями был острым.
В этом случае ковариантная (измеренная) координата $i_{\scriptscriptstyle(-A)}$ на оси тока  $I_{\scriptscriptstyle(-A)}$ равнa контравариантной (неизмеряемой) 
координате $i^{\scriptscriptstyle(-A)}$ на сопряженной %к $i_{\scriptscriptstyle(-A)}$  
оси~$I^{\scriptscriptstyle(-A)}$, которая сонаправлена с $U_{\scriptscriptstyle AB}$. 
Контравариантная (неизмеряемая) координата $u^{\scriptscriptstyle AB }$
по сопряженной оси, которая параллельна оси $I_{(-A)}$, равна ковариантной (измеренной) координате $u_{\scriptscriptstyle AB}$.
\begin{figure}[!ht]
\centering
	\begin{circuitikz}[scale=0.8]
        \newcommand{\Axis}{6.3}
        \newcommand{\Axisy}{4.5}
        \newcommand{\Axisyy}{-1.4}
        \newcommand{\gammaa}{120} % угол между осями
        \newcommand{\E}{2.3}
        \newcommand{\alfa}{10} % угол вектора i
	\newcommand{\betaa}{23} % угол вектора u
        \newcommand{\V}{4.9}
	\newcommand{\UU}{4.1}

        \draw[thin,->] (0,0) -- ({\Axis*cos(0)},{\Axis*sin(0)}) node [right] {$U_{\scriptscriptstyle  AB}, I^{\scriptscriptstyle  AB}$};
        \draw[thin,->] ({\Axisyy*cos(\gammaa)},{\Axisyy*sin(\gammaa)}) -- ({\Axisy*cos(\gammaa)},{\Axisy*sin(\gammaa)}) node [above] {$U_{\scriptscriptstyle  BC}, I^{\scriptscriptstyle  BC}$};

        % сопряженные оси
	\draw[thin,->] (0,0) -- ({\Axis*cos(\gammaa-90)},{\Axis*sin(\gammaa-90)}) node [right] {$I_{\scriptscriptstyle  (-A)}$};
        \draw[thin,->] ({\Axisyy*cos(90)},{\Axisyy*sin(90)}) -- ({\Axisy*cos(90)},{\Axisy*sin(90)}) node [left] {$I_{\scriptscriptstyle  C}$};

       % сам вектор
        \draw[thick, red,->] (0,0) -- ({\V*cos(\alfa)},{\V*sin(\alfa)}) node (V) {} node[above right] {$\vec{i}$};
	\draw[thick, blue,->] (0,0) -- ({\UU*cos(\betaa)},{\UU*sin(\betaa)}) node (U) {} node[right] {$\vec{u}$};
        % перпендикулярные проекции
	\draw[dashed] ({\UU*cos(\betaa)},{\UU*sin(\betaa)}) to[short,-] ({\UU*cos(\betaa)},0) node[below] {$u_{\scriptscriptstyle AB}$};
        \filldraw[color=white, draw=black] ({\UU*cos(\betaa)},0)  circle (0.56mm);

       %проекция на сопряженную ось e^{AB}
	\newcommand{\VAB}{\V*cos(\gammaa-90-\alfa)} % проекция на ось V^AB
        \draw[dashed]  (V.center) -- ({\VAB/cos(\gammaa-90)}, 0) node[below] {$i^{\scriptscriptstyle AB}$};
        \filldraw[color=white, draw=black] ({\VAB/cos(\gammaa-90)}, 0)  circle (0.56mm);

	\draw[dashed]  (V.center) -- ({\VAB*cos(\gammaa-90)}, {\VAB*sin(\gammaa-90)}) node[above=1.3mm] {$i_{\scriptscriptstyle (-A)}$};
        \filldraw[color=white, draw=black]  ({\VAB*cos(\gammaa-90)}, {\VAB*sin(\gammaa-90)})  circle (0.56mm);



        \newcommand{\Ubc}{(\UU*cos(\betaa)*cos(\gammaa) + \UU*sin(\betaa)*sin(\gammaa))} % проекция на ось Vbc
        \draw[dashed] (U.center) -- ({\Ubc*cos(\gammaa)},{\Ubc*sin(\gammaa)}) node[right=1.8mm] {$u^{\scriptscriptstyle BC}$};
        \filldraw[color=white, draw=black]  ({\Ubc*cos(\gammaa)},{\Ubc*sin(\gammaa)}) circle (0.56mm);

        % проекции вдоль оси X_1
        \draw[dashed]  (V.center) -- ({-\V*sin(\alfa)*tan(\gammaa-90)} , {\V*sin(\alfa)}) node[left] {$i^{\scriptscriptstyle BC}$};
        \filldraw[color=white, draw=black]  ({-\V*sin(\alfa)*tan(\gammaa-90)} , {\V*sin(\alfa)}) circle (0.56mm);
        \draw[dashed]  (V.center) -- (0, {\V*sin(\alfa)}) node[above right] {$i_{\scriptscriptstyle C}$};
        \filldraw[color=white, draw=black]  (0, {\V*sin(\alfa)}) circle (0.56mm);

\end{circuitikz}
	\caption{выбор осей измерений линейных напряжений и фазных токов}
        \label{pickup_mesure}
\end{figure}
При таком выборе осей мгновенная мощность $p$, которая равна скалярному произведению (\ref{scalar}) изображающих векторов тока и напряжения:
$$
p = (\vec{i}\cdot\vec{u}) = u_{\scriptscriptstyle AB} i^{\scriptscriptstyle AB} + u_{\scriptscriptstyle BC} i^{\scriptscriptstyle BC}
%u_{\scriptscriptstyle AB} i_{\scriptscriptstyle (-A)} + u_{\scriptscriptstyle BC} i_{\scriptscriptstyle C}
$$
Учитывая $i_{\scriptscriptstyle A} + i_{\scriptscriptstyle B} +  i_{\scriptscriptstyle C} = 0$, введя «нулевой» потенциал~$v_0$ фазного напряжения
(«нулевой» потенциал может быть функцией от времени $v_0(t)$)
$$
p = (u_{\scriptscriptstyle B} - v_{\scriptscriptstyle 0} +  v_{\scriptscriptstyle 0} - u_{\scriptscriptstyle A}) i_{\scriptscriptstyle (-A)} +  
(u_{\scriptscriptstyle C} - v_{\scriptscriptstyle 0} +  v_{\scriptscriptstyle 0} - u_{\scriptscriptstyle B}) i_{\scriptscriptstyle C} =
$$
$$
 = u_{\scriptscriptstyle A} i_{\scriptscriptstyle A} + u_{\scriptscriptstyle B} i_{\scriptscriptstyle B}  + u_{\scriptscriptstyle C} i_{\scriptscriptstyle C} 
$$
что является суммой мощностей фаз. 

Наиболее часто в системах векторного управления пользуются выражением (\ref{park-gorev}) и переходят от трехфазной системы токов и напряжений сначала к 
стационарной декартовой системе, а затем к вращающейся системе с осью вдоль одного из изображающих векторов (d-) и квадратурной перпендикулярной осью (q-).
После введения управления с учетом обратных связей переходят обратно к трехфазной системе.

Предъявим метод нахождения компонент (d-) и (q-):

Линейная комбинация $\alpha (u_{\scriptscriptstyle AB}, u_{\scriptscriptstyle BC})$ -- это компонента вдоль оси (d-).

Скалярное произведение вектора с координатами $(-u_{\scriptscriptstyle BC}, u_{\scriptscriptstyle AB})$ и исходного вектора:
$$
-u_{\scriptscriptstyle BC} u_{\scriptscriptstyle AB} + u_{\scriptscriptstyle AB} u_{\scriptscriptstyle BC} = 0,
$$ 
вектор с координатами $(-u_{\scriptscriptstyle BC}, u_{\scriptscriptstyle AB})$ перпендикулярен исходному и, значит, это компонента вдоль квадратурной оси (q-).
Обе компоненты (d-) и (q-) оказываются выражены через изменяемые величины.

Благодаря этому методу возможно построить систему управления электрической машины без перехода в декартову систему и обратно, что позволяет существенно сократить объем вычислений.
\begin{thebibliography}{4}
       \bibitem{Gorev}Горев А.А. Переходные процессы синхронной машины. -- М.,Л., Гос. энергетическое изд., 1950. -- 551 c.
        \bibitem{Sokolovsky}Соколовский Г.Г. Электроприводы переменного тока с частотным регулированием: Учебник для студ. высш.учеб.заведений.
                -- М. «Академия», 2007 - 272 с.
	\bibitem{Borisenko}Борисенко А.И., Тарапов И.Е. Векторный анализ и начала тензорного исчисления. -- М. «Высшая школа», 1966. -- 252 с.
	\bibitem{Prokshin}Илюшин А.Г., Маслов И.А., Прокшин А.Н. и др. О системах координат для математического описания систем управления электропривода. --
		Сборник докладов 71-й научно-технической конференции ППС, СПб, 2018, с.172
\end{thebibliography}
% \bibitem{}
\end{document}
