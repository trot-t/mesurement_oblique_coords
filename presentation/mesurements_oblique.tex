\providecommand{\pdfxopts}{a-1b,cyrxmp}
\providecommand{\thisyear}{2021}
\immediate\write18{rm \jobname.xmpdata}%  uncomment for Unix-based systems
\begin{filecontents*}{\jobname.xmpdata}
	\Title{Измерение тока и напряжения в косоугольных координатах в трехфазной обобщенной электрической машине \textemdash\thisyear}
\Author{
Альмушреки Осама Абду Али\sep
Обама Омбеде Никола Серж\sep
Прокшин Артем Николаевич\sep
Татаринцев Николай Иванович\sep
Трофимов Александр Викторович}
\Creator{pdfTeX + pdfx.sty with options \pdfxopts }
\Subject{Измерение тока и напряжения в косоугольных координатах в трехфазной обобщенной электрической машине}
\Keywords{измерение, фазный ток, линейное напряжение, ковариантная координата, контравариантная координата}
\CoverDisplayDate{март \thisyear}
\CoverDate{2021-05-22}
\Copyrighted{True}
\Copyright{Public Domain}
\CopyrightURL{http://github.com/trot-t}
\Creator{pdfTeX + pdfx.sty with options \pdfxopts }
\end{filecontents*}

\documentclass[14pt]{beamer}

\pdfcompresslevel=9

\usepackage[\pdfxopts]{pdfx}[2016/03/09]
\PassOptionsToPackage{obeyspaces}{url}
\let\tldocrussian=1  % for live4ht.cfg

\usepackage[T2A]{fontenc}
\usepackage[utf8]{inputenc}
\usepackage[english,russian]{babel}
\usepackage{booktabs}
\usepackage{tikz}
\usepackage[european,cuteinductors,smartlabels]{circuitikz}
\usetikzlibrary{arrows.meta, shadows}

\usepackage{amssymb,amsfonts,amsmath,mathtext}
\usepackage{amssymb}
\usepackage{cite,enumerate,float,indentfirst}
\usepackage{cancel}
\usepackage{csquotes}
\newcommand{\quotes}[1]{``#1''}
\usetikzlibrary{calc}

\usepackage{advdate}

%\usepackage{pgfplots}
%\usepackage[left=1cm,right=1cm, top=1cm,bottom=1cm,bindingoffset=0cm]{geometry}

% Beamer — верстаем презентации  https://habrahabr.ru/post/145523/ 
\graphicspath{{images/}}

\usetheme{Pittsburgh}
\usecolortheme{whale}

\setbeamercolor{footline}{fg=blue}
\setbeamertemplate{footline}{
\leavevmode%
\hbox{%
\begin{beamercolorbox}[wd=.333333\paperwidth,ht=2.25ex,dp=1ex,center]{}%
Прокшин А.Н. и др.
\end{beamercolorbox}%
\begin{beamercolorbox}[wd=.333333\paperwidth,ht=2.25ex,dp=1ex,center]{}%
Санкт-Петербург, 2021
\end{beamercolorbox}%
\begin{beamercolorbox}[wd=.333333\paperwidth,ht=2.25ex,dp=1ex,right]{}%
Стр. \insertframenumber{} из \inserttotalframenumber \hspace*{2ex}
\end{beamercolorbox}}%
\vskip0pt%
}

\newcommand{\itemi}{\item[\checkmark]}

	\usefonttheme[onlymath]{serif} % в формулах использовать текст с засечками
\begin{document}
\title{\small{Измерение тока и напряжения в косоугольных координатах в трехфазной обобщенной электрической машине }}
\author{\small{%
\emph{авторы:}~Альмушреки Осама Абду Али\\
\emph{}~Обама Омбеде Никола Серж\\
\emph{}~Прокшин Артем Николаевич\\%
\emph{}~Татаринцев Николай Иванович\\
\emph{}~Трофимов Александр Викторович}}



\institute{Санкт-Петербургский государственный электротехнический университет «ЛЭТИ» им. В.И. Ульянова (Ленина)}
\vspace{30pt}%

\vspace{60pt}%

\AdvanceDate[-1] % печатаю в субботу а нужна пятница
%\date{\small{Санкт-Петербург, 2021}}

\AtBeginSection{
	\begin{frame}
		\frametitle{Содержание}
		\tableofcontents[currentsection]
	\end{frame}
}

\begin{frame}
\titlepage	
\end{frame}


\begin{frame}
\frametitle{\small инвертор напряжения синхронизированный с сетью}
	\vspace{-1cm}
\begin{figure}[!ht]
\begin{circuitikz}[scale=1]
\ctikzset{bipoles/length=1.0cm}
%\ctikzset{iloop /.style={width=2.0}}
\draw
(1.25,2.65)node[nigbt,bodydiode](npn1){}% 1 ряд 
%\node[nigbt,bodydiode] (npn1) at (1.25,3.1) {};% 1 ряд 
(1.25,.55) node[nigbt,bodydiode](npn4){}%1ряд
(npn1.S) -- (npn4.D);

% найдем положение плюсовой шины
\path let \p1 = (npn1.D) in node(plus)  at (0,\y1) {};
\draw (0,0) to[C] (plus)

(2.75,2.65)node[nigbt,bodydiode](npn3){}% 2 ряд 
(2.75,.55) node[nigbt,bodydiode](npn6){}% 2ряд
(npn3.S) -- (npn6.D)

(4.25,2.65)node[nigbt,bodydiode](npn5){}%последний ряд 
(4.25,.55) node[nigbt,bodydiode](npn2){}%последний ряд
(npn5.S) -- (npn2.D)

(plus.center) --(npn1.D) node[above]{} -- (npn3.D) node[above]{} -- (npn5.D) node[above]{} % плюсовая шина
(0,0) -- (npn4.S) node[below]{} -- (npn6.S) node[below]{} -- (npn2.S) node[below]{} % минусовая шина

($(npn5.S)!0.85!(npn2.D)$)   node[left]{\scriptsize$C$} to[short,-] ++ (0.5,0) to[iloop, name=Ic, mirror] ++(0.5,0) to[L,american inductor,-] ++ (1.5,0) --++ (1.5,0)  node(VC) {} 
	 --++ (.5,0) node(C) {}
	($(npn3.S)!0.5!(npn6.D)$) node[left]{\scriptsize$B$} to[short,-] ($(npn5.S)!0.5!(npn2.D)$) --++ (1.0,0) to[L,american inductor,-] ++ (1.5,0) --++(.5,0) node(VB) {} --++ (.5,0) node(VB1){} 
	--++ (1,0) % катуха В 
	($(npn1.S)!0.15!(npn4.D)$) node[left]{\scriptsize$A$}  to[iloop, name=Ia, mirror,-] ++(1.5,0) to[short] ($(npn5.S)!0.15!(npn2.D)$) --++ (1,0) to[L,american inductor,-] 
	++ (1.5,0) node(VA) {} --++ (2,0) node(A) {}

(A.center)--(C.center) (A.center) --++(0,1.5) (C.center) --++ (0,-1.5)

(A.center) ++(0.1,0.8) --++ (-0.2,0.2)
(A.center) ++(0.1,0.9) --++ (-0.2,0.2)
(A.center) ++(0.1,1.0) --++ (-0.2,0.2)

(Ia) --++ (0,-2.2) node[below] {\small $i_{\scriptscriptstyle A}$}
(Ic) --++ (0,-1.7) node[below] {\small $i_{\scriptscriptstyle C}$}

%(VB.center) --++(0,-1) to[smeter,t=v,l_=$\small v_{\scriptscriptstyle BC}$] ++(1,0) -- (VC.center)
(VB.center) --++(0,-1) to[rmeterwa,t=v,l_=$\small u_{\scriptscriptstyle BC}$] ++(1,0) -- (VC.center)
(VA.center)  --++(0,1) to[rmeterwa,t=v,l^=$\small u_{\scriptscriptstyle AB}$] ++(1,0) -- (VB1.center);
%(VA.center)  --++(0,1) to[smeter,t=v,l_=$\small v_{\scriptscriptstyle AB}$] ++(1,0) -- (VB1.center);
        \draw[<-] (4.9,3.9) -- (6.5,3.9);
	\draw[->] (4.9,4.1) -- (6.5,4.1) node[midway, above] {$p,q$};
;\end{circuitikz}

%        \caption{схема инвертора напряжения ведомого сетью и изображающий вектор напряжения инвертора}
%        \label{invertor_with_grid}
\end{figure}
\end{frame}


\begin{frame}
	\frametitle{\small измеряемые и математические координаты вектора напряжения}
\vspace{-0.2cm}
\hspace{-1.2cm}
\begin{tabular}{cl}
\begin{minipage}[h]{0.49\linewidth}

\begin{figure}[!ht]
\caption{измеряемые координаты %изображающего вектора 
	-- ковариантные} 
\centering
\begin{circuitikz}
        \newcommand{\Axis}{4.3}
        \newcommand{\Axisy}{2.4}
        \newcommand{\Axisyy}{-0.9}
        \newcommand{\gammaa}{120} % угол между осями
        \newcommand{\E}{2.7}
        \newcommand{\alfa}{20} % угол вектора
        \newcommand{\V}{3.5}
        \draw[thin,->] (0,0) -- ({\Axis*cos(0)},{\Axis*sin(0)}) node [below] {$U_{\scriptscriptstyle AB}$};
        \draw[thin,->] ({\Axisyy*cos(\gammaa)},{\Axisyy*sin(\gammaa)}) -- ({\Axisy*cos(\gammaa)},{\Axisy*sin(\gammaa)}) node [above right] {$U_{\scriptscriptstyle BC}$};
%        \draw[thick, ->] (0,0) -- ({\E*cos(0)},{\E*sin(0)}) node[below] {$\vec{e}_{\scriptscriptstyle AB}$};
%        \draw[thick, ->] (0,0) -- ({\E*cos(\gammaa)},{\E*sin(\gammaa)}) node[left] {$\vec{e}_{\scriptscriptstyle BC}$};
        % сам вектор
        \draw[blue,->] (0,0) -- ({\V*cos(\alfa)},{\V*sin(\alfa)}) node (V) {} node[right] {$\vec{u}$};
        % перпендикулярные проекции
        \draw[dashed] ({\V*cos(\alfa)},{\V*sin(\alfa)}) -- ({\V*cos(\alfa)},0) node[below] {$u_{\scriptscriptstyle\!A\!B}$};
        \newcommand{\Vbc}{(\V*cos(\alfa)*cos(\gammaa) + \V*sin(\alfa)*sin(\gammaa))} % проекция на ось Vbc
%       \draw[dashed] (V.center) -- ({\Vbc*cos(\gammaa)},{\Vbc*sin(\gammaa)}) node[below left=-1mm] {$v_{\scriptscriptstyle\!B\!C}$};
        \draw[dashed] (V.center) -- ({\Vbc*cos(\gammaa)},{\Vbc*sin(\gammaa)}) node[left=1mm] {$u_{\scriptscriptstyle\!B\!C}$};

        %проекция на сопряженную ось e^{AB}
        \newcommand{\VAB}{(\V*cos(\alfa)*cos(\gammaa-90-\alfa) + \V*sin(\alfa)*sin(\gammaa-90-\alfa))} % проекция на ось V^AB
%        \draw[dashed]  (V.center) -- ({\VAB/cos(\gammaa-90)}, 0) node[below] {$u^{\scriptscriptstyle\!A\!B}$};
        %проекция на сопряженную ось e^{BC}
        \newcommand{\VBC}{\V*sin(\alfa)}
%        \draw[dashed]  (V.center) --  ({\VBC/cos(\gammaa-90)*cos(\gammaa)}, {\VBC/cos(\gammaa-90)*sin(\gammaa)}) node[below left=-1mm] {$u^{\scriptscriptstyle\!B\!C}$};
\end{circuitikz}

\end{figure}
\end{minipage}

\begin{minipage}[h]{0.50\linewidth}

\begin{figure}[!ht]
\caption{математические координаты %изображающего вектора 
	-- контравариантные}
\centering

\begin{circuitikz}
        \newcommand{\Axis}{4.7}
        \newcommand{\Axisy}{2.4}
        \newcommand{\Axisyy}{-0.9}
        \newcommand{\gammaa}{120} % угол между осями
        \newcommand{\E}{2.7}
        \newcommand{\alfa}{20} % угол вектора
        \newcommand{\V}{3.5}
        \draw[thin,->] (0,0) -- ({\Axis*cos(0)},{\Axis*sin(0)}) node [above right] {$U_{\scriptscriptstyle AB}$};
        \draw[thin,->] ({\Axisyy*cos(\gammaa)},{\Axisyy*sin(\gammaa)}) -- ({\Axisy*cos(\gammaa)},{\Axisy*sin(\gammaa)}) node [above right] {$U_{\scriptscriptstyle BC}$};
%        \draw[thick, ->] (0,0) -- ({\E*cos(0)},{\E*sin(0)}) node[below] {$\vec{e}_{\scriptscriptstyle AB}$};
%        \draw[thick, ->] (0,0) -- ({\E*cos(\gammaa)},{\E*sin(\gammaa)}) node[left] {$\vec{e}_{\scriptscriptstyle BC}$};
        % сам вектор
        \draw[blue,->] (0,0) -- ({\V*cos(\alfa)},{\V*sin(\alfa)}) node (V) {} node[right] {$\vec{u}$};
        % перпендикулярные проекции
%        \draw[dashed] ({\V*cos(\alfa)},{\V*sin(\alfa)}) -- ({\V*cos(\alfa)},0) node[below] {$u_{\scriptscriptstyle\!A\!B}$};
        \newcommand{\Vbc}{(\V*cos(\alfa)*cos(\gammaa) + \V*sin(\alfa)*sin(\gammaa))} % проекция на ось Vbc
%       \draw[dashed] (V.center) -- ({\Vbc*cos(\gammaa)},{\Vbc*sin(\gammaa)}) node[below left=-1mm] {$v_{\scriptscriptstyle\!B\!C}$};
%        \draw[dashed] (V.center) -- ({\Vbc*cos(\gammaa)},{\Vbc*sin(\gammaa)}) node[left=1mm] {$u_{\scriptscriptstyle\!B\!C}$};

        %проекция на сопряженную ось e^{AB}
        \newcommand{\VAB}{(\V*cos(\alfa)*cos(\gammaa-90-\alfa) + \V*sin(\alfa)*sin(\gammaa-90-\alfa))} % проекция на ось V^AB
        \draw[dashed]  (V.center) -- ({\VAB/cos(\gammaa-90)}, 0) node[below] {$u^{\scriptscriptstyle\!A\!B}$};
        %проекция на сопряженную ось e^{BC}
        \newcommand{\VBC}{\V*sin(\alfa)}
        \draw[dashed]  (V.center) --  ({\VBC/cos(\gammaa-90)*cos(\gammaa)}, {\VBC/cos(\gammaa-90)*sin(\gammaa)}) node[above right=-1mm] {$u^{\scriptscriptstyle\!B\!C}$};
\end{circuitikz}
        
\end{figure}
\end{minipage}

\end{tabular}
\vspace{0.5cm}

$$
|\vec{u}|^2 = (\vec{u} \cdot \vec{u})  = u_{\scriptscriptstyle AB}u^{\scriptscriptstyle AB} +  u_{\scriptscriptstyle BC}u^{\scriptscriptstyle BC}
$$
\end{frame}


\begin{frame}
\frametitle{\small в симметричной системе известное выражение}
$$
	\vec{i} = \frac{2}{3}\left(\vec{i}_{a_{\!\perp}} + \vec{i}_{b_{\!\perp}} + \vec{i}_{c_{\!\perp}}\right) 
$$
%
\begin{tabular}{cl}
\begin{minipage}[h]{0.3\linewidth}
\begin{tikzpicture}[scale=2]
\newcommand{\D}){8}
\draw[->, very thin,>=latex] (0,0) -- ({cos(-30)}, {sin(-30)}) node[below] {$\scriptstyle I_B$};
\draw[->, very thin,>=latex] (0,0) -- ({cos(90)},{sin(90)})  node[above] {$\scriptstyle I_C$};
\draw[->, very thin,>=latex] (0,0) -- ({cos(210)},{sin(210)}) node[below] {$\scriptstyle I_A$};

\draw[green, very thick,->,>=latex] (0,0) -- ( {0.59*cos(-30) - 0*sin(-30)}, {0.59*sin(-30) + 0*cos(-30)});
\draw[red, very thick,->,>=latex] (0,0) -- ({(-0.20)*cos(-30) - 0.35*sin(-30)}, {(-0.20)*sin(-30) +  0.35*cos(-30)});
\draw[yellow, very thick,->,>=latex] (0,0) -- ({0.50*cos(-30) - 0.86*sin(-30)} ,{0.50*sin(-30) + 0.86*cos(-30)} );
\draw[->,thick] (0,0) -- ({0.59*cos(-30) -  0.81*sin(-30)}, {0.59*sin(-30) +  0.81*cos(-30)}) node[below] {$\vec{i}$};
\end{tikzpicture} 
\end{minipage}
&
\begin{minipage}[h]{0.7\linewidth}
	{\small\begin{itemize}
\item измеряемые величины -- перпендикулярные координаты вектора % \cite{Gorev},\cite{Sokolovsky}; 
%\item физическая величина -- всегда произведение  ко- и контра-вариантных координат;
%\item изображающий вектор симметричной системы есть $2/3 (I_a + I_b + I_c)$;
%\item рисунок -- есть результат 1й практической работы.
\end{itemize}
	}
\end{minipage}
\end{tabular}
\end{frame}


\begin{frame}
\frametitle{\small координаты %вектора 
	в двойственных (сопряженных) базисах}
\begin{figure}[!ht]
\centering
\begin{circuitikz}
        \newcommand{\Axis}{6.3}
        \newcommand{\Axisy}{4.5}
        \newcommand{\Axisyy}{-1.4}
        \newcommand{\gammaa}{120} % угол между осями
        \newcommand{\E}{3.2}
        \newcommand{\alfa}{20} % угол вектора
        \newcommand{\V}{4.5}
        \draw[thin,->] (0,0) -- ({\Axis*cos(0)},{\Axis*sin(0)}) node [right] {$\small X_{\scriptstyle 1}$};
        \draw[thin,->] ({\Axisyy*cos(\gammaa)},{\Axisyy*sin(\gammaa)}) -- ({\Axisy*cos(\gammaa)},{\Axisy*sin(\gammaa)}) node [above] {$\small X_{\scriptstyle 2}$};
        \draw[thick, ->] (0,0) -- ({\E*cos(0)},{\E*sin(0)}) node[below] {$\vec{e}_{\scriptstyle 1}$};
        \draw[thick, ->] (0,0) -- ({\E*cos(\gammaa)},{\E*sin(\gammaa)}) node[left] {$\vec{e}_{\scriptstyle 2}$};
        % сопряженные оси
        \draw[thin,->] (0,0) -- ({\Axis*cos(\gammaa-90)},{\Axis*sin(\gammaa-90)}) node [right] {$\small X^{\scriptstyle 1}$};
        \draw[thin,->] ({\Axisyy*cos(90)},{\Axisyy*sin(90)}) -- ({\Axisy*cos(90)},{\Axisy*sin(90)}) node [above] {$\small X^{\scriptstyle 2}$};
        \draw[thick, ->] (0,0) -- ({\E*cos(0)},{\E*tan(\gammaa-90)}) node[above left=-1.5mm] {$\vec{e}^{\scriptstyle\;1}$};
        \draw[thick, ->] (0,0) -- (0, {\E/cos(\gammaa-90)}) node[right] {$\vec{e}^{\scriptstyle\;2}$};

       % сам вектор
        \draw[thick, blue] (0,0) -- ({\V*cos(\alfa)},{\V*sin(\alfa)}) node (V) {} node[right] {$\vec{x}$};
        % перпендикулярные проекции
        \draw[dashed] ({\V*cos(\alfa)},{\V*sin(\alfa)}) to[short,-] ({\V*cos(\alfa)},0) node[below] {$\frac{x_{\scriptscriptstyle 1}}{|e_{\scriptscriptstyle 1}|}$};
        \filldraw[color=white, draw=black] ({\V*cos(\alfa)},0)  circle (0.56mm);
        % продолжим до сопряженной оси
        \draw[dashed] ({\V*cos(\alfa)},{\V*sin(\alfa)}) to[short] ({\V*cos(\alfa)},{\V*cos(\alfa)*tan(\gammaa-90)}) node[right] {${\scriptstyle x_{\scriptscriptstyle 1}|e^{\scriptscriptstyle 1}|}$};
        \filldraw[color=white, draw=black] ({\V*cos(\alfa)},{\V*cos(\alfa)*tan(\gammaa-90)}) circle (0.56mm);

        \newcommand{\Vbc}{(\V*cos(\alfa)*cos(\gammaa) + \V*sin(\alfa)*sin(\gammaa))} % проекция на ось Vbc
        % продолжим до вертикальной оси
        \draw[dashed] (V.center) -- (0,{\Vbc/cos(\gammaa-90)}) node[left] {${\scriptstyle x_{\scriptscriptstyle 2} |e^{\scriptscriptstyle 2}|}$};
        \filldraw[color=white, draw=black]  (0,{\Vbc/cos(\gammaa-90)})   circle (0.56mm);
        \draw[dashed] (V.center) -- ({\Vbc*cos(\gammaa)},{\Vbc*sin(\gammaa)}) node[right=1.8mm] {$\frac{x^{\scriptscriptstyle 2}}{|e^{\scriptscriptstyle 2}|}$};
        \filldraw[color=white, draw=black]  ({\Vbc*cos(\gammaa)},{\Vbc*sin(\gammaa)}) circle (0.56mm);

        %проекция на сопряженную ось e^{AB}
%        \newcommand{\VAB}{(\V*cos(\alfa)*cos(\gammaa-90-\alfa) + \V*sin(\alfa)*sin(\gammaa-90-\alfa))} % проекция на ось V^AB  -- clamsy error
        \newcommand{\VAB}{\V*cos(\gammaa-90-\alfa)} % проекция на ось V^AB
        \draw[dashed]  (V.center) -- ({\VAB/cos(\gammaa-90)}, 0) node[below] {${\scriptstyle x^{\scriptscriptstyle 1} |e_{\scriptscriptstyle 1}|}$};
        \filldraw[color=white, draw=black] ({\VAB/cos(\gammaa-90)}, 0)  circle (0.56mm);
        \draw[dashed]  (V.center) -- ({\VAB*cos(\gammaa-90)}, {\VAB*sin(\gammaa-90)}) node[above left=-1.4mm] {$\frac{x^{\scriptscriptstyle 1}}{|e^{\scriptscriptstyle 1}|} $};
        \filldraw[color=white, draw=black]  ({\VAB*cos(\gammaa-90)}, {\VAB*sin(\gammaa-90)})  circle (0.56mm);
        % проекции вдоль оси X_1
        \draw[dashed]  (V.center) -- ({-\V*sin(\alfa)*tan(\gammaa-90)} , {\V*sin(\alfa)}) node[left] {${\scriptstyle x^{\scriptscriptstyle 2} |e_{\scriptscriptstyle 2}|}$};
        \filldraw[color=white, draw=black]  ({-\V*sin(\alfa)*tan(\gammaa-90)} , {\V*sin(\alfa)}) circle (0.56mm);
        \draw[dashed]  (V.center) -- (0, {\V*sin(\alfa)}) node[above right] {$\frac{x^{\scriptscriptstyle 2}}{|e^{\scriptscriptstyle 2}|} $};
        \filldraw[color=white, draw=black]  (0, {\V*sin(\alfa)}) circle (0.56mm);

\end{circuitikz}
\end{figure}
\end{frame}

\begin{frame}
\frametitle{\small выбор осей измерений линейных напряжений и фазных токов}
%\begin{figure}[!ht]
%\centering
\begin{circuitikz}
        \newcommand{\Axis}{6.0}
        \newcommand{\Axisy}{4.0}
        \newcommand{\Axisyy}{-1.4}
        \newcommand{\gammaa}{120} % угол между осями
        \newcommand{\E}{2.3}
        \newcommand{\alfa}{10} % угол вектора i
        \newcommand{\betaa}{23} % угол вектора u
        \newcommand{\V}{4.9}
        \newcommand{\UU}{4.1}

        \draw[thin,->] (0,0) -- ({\Axis*cos(0)},{\Axis*sin(0)}) node [right] {$U_{\scriptscriptstyle  AB}, I^{\scriptscriptstyle  AB}$};
        \draw[thin,->] ({\Axisyy*cos(\gammaa)},{\Axisyy*sin(\gammaa)}) -- ({\Axisy*cos(\gammaa)},{\Axisy*sin(\gammaa)}) node [above] {$U_{\scriptscriptstyle  BC}, I^{\scriptscriptstyle  BC}$};

        % сопряженные оси
        \draw[thin,->] (0,0) -- ({\Axis*cos(\gammaa-90)},{\Axis*sin(\gammaa-90)}) node [right] {$I_{\scriptscriptstyle  (-A)}$};
        \draw[thin,->] ({\Axisyy*cos(90)},{\Axisyy*sin(90)}) -- ({\Axisy*cos(90)},{\Axisy*sin(90)}) node [right] {$I_{\scriptscriptstyle  C}$};

       % сам вектор
        \draw[thick, red,->] (0,0) -- ({\V*cos(\alfa)},{\V*sin(\alfa)}) node (V) {} node[above right] {$\vec{i}$};
        \draw[thick, blue,->] (0,0) -- ({\UU*cos(\betaa)},{\UU*sin(\betaa)}) node (U) {} node[right] {$\vec{u}$};
        % перпендикулярные проекции
        \draw[dashed] ({\UU*cos(\betaa)},{\UU*sin(\betaa)}) to[short,-] ({\UU*cos(\betaa)},0) node[below] {$u_{\scriptscriptstyle AB}$};
        \filldraw[color=white, draw=black] ({\UU*cos(\betaa)},0)  circle (0.56mm);

       %проекция на сопряженную ось e^{AB}
        \newcommand{\VAB}{\V*cos(\gammaa-90-\alfa)} % проекция на ось V^AB
        \draw[dashed]  (V.center) -- ({\VAB/cos(\gammaa-90)}, 0) node[below] {$i^{\scriptscriptstyle AB}$};
        \filldraw[color=white, draw=black] ({\VAB/cos(\gammaa-90)}, 0)  circle (0.56mm);

        \draw[dashed]  (V.center) -- ({\VAB*cos(\gammaa-90)}, {\VAB*sin(\gammaa-90)}) node[above=1.3mm] {$i_{\scriptscriptstyle (-A)}$};
        \filldraw[color=white, draw=black]  ({\VAB*cos(\gammaa-90)}, {\VAB*sin(\gammaa-90)})  circle (0.56mm);



        \newcommand{\Ubc}{(\UU*cos(\betaa)*cos(\gammaa) + \UU*sin(\betaa)*sin(\gammaa))} % проекция на ось Vbc
        \draw[dashed] (U.center) -- ({\Ubc*cos(\gammaa)},{\Ubc*sin(\gammaa)}) node[right=1.8mm] {$u^{\scriptscriptstyle BC}$};
        \filldraw[color=white, draw=black]  ({\Ubc*cos(\gammaa)},{\Ubc*sin(\gammaa)}) circle (0.56mm);

        % проекции вдоль оси X_1
        \draw[dashed]  (V.center) -- ({-\V*sin(\alfa)*tan(\gammaa-90)} , {\V*sin(\alfa)}) node[left] {$i^{\scriptscriptstyle BC}$};
        \filldraw[color=white, draw=black]  ({-\V*sin(\alfa)*tan(\gammaa-90)} , {\V*sin(\alfa)}) circle (0.56mm);
        \draw[dashed]  (V.center) -- (0, {\V*sin(\alfa)}) node[above right] {$i_{\scriptscriptstyle C}$};
        \filldraw[color=white, draw=black]  (0, {\V*sin(\alfa)}) circle (0.56mm);

\end{circuitikz}
%        \caption{выбор осей измерений линейных напряжений и фазных токов}
%        \label{pickup_mesure}
%\end{figure}
$$
p = (\vec{i}\cdot\vec{u}) = u_{\scriptscriptstyle AB} i^{\scriptscriptstyle AB} + u_{\scriptscriptstyle BC} i^{\scriptscriptstyle BC}
%u_{\scriptscriptstyle AB} i_{\scriptscriptstyle (-A)} + u_{\scriptscriptstyle BC} i_{\scriptscriptstyle C}
$$
%\vspace{-0.7cm}\hspace{-1.2cm}
$$
	\scriptstyle{
p = (u_{\scriptscriptstyle B} - v_{\scriptscriptstyle 0} +  v_{\scriptscriptstyle 0} - u_{\scriptscriptstyle A}) i_{\scriptscriptstyle (-A)} +
(u_{\scriptscriptstyle C} - v_{\scriptscriptstyle 0} +  v_{\scriptscriptstyle 0} - u_{\scriptscriptstyle B}) i_{\scriptscriptstyle C} =
 u_{\scriptscriptstyle A} i_{\scriptscriptstyle A} + u_{\scriptscriptstyle B} i_{\scriptscriptstyle B}  + u_{\scriptscriptstyle C} i_{\scriptscriptstyle C}
}
$$
\end{frame}

\begin{frame}
\frametitle{\smallподход}
Предъявим $d-$, $q-$:
	\begin{itemize}
		\item $d-\;$  линейная комбинация $\alpha \cdot \vec{u}$;

		\item $q-\;$  $\vec{i} = (i_{\scriptscriptstyle A}, i_{\scriptscriptstyle C} ) \Rightarrow (-i_{\scriptscriptstyle C}, i_{\scriptscriptstyle A} )$
	\end{itemize}
	\vspace{0.5cm}

\begin{circuitikz}
        \newcommand{\Axis}{6.0}
        \newcommand{\Axisy}{3.8}
        \newcommand{\Axisyy}{-0.8}
        \newcommand{\gammaa}{120} % угол между осями
        \newcommand{\E}{2.3}
        \newcommand{\alfa}{10} % угол вектора i
        \newcommand{\betaa}{23} % угол вектора u
        \newcommand{\V}{4.9}
        \newcommand{\UU}{4.1}

        \draw[thin,->] (0,0) -- ({\Axis*cos(0)},{\Axis*sin(0)}) node [right] {$U_{\scriptscriptstyle  AB}, I^{\scriptscriptstyle  AB}$};
        \draw[thin,->] ({\Axisyy*cos(\gammaa)},{\Axisyy*sin(\gammaa)}) -- ({\Axisy*cos(\gammaa)},{\Axisy*sin(\gammaa)}) node [above] {$U_{\scriptscriptstyle  BC}, I^{\scriptscriptstyle  BC}$};

        % сопряженные оси
        \draw[thin,->] (0,0) -- ({\Axis*cos(\gammaa-90)},{\Axis*sin(\gammaa-90)}) node [right] {$I_{\scriptscriptstyle  (-A)}$};
        \draw[thin,->] ({\Axisyy*cos(90)},{\Axisyy*sin(90)}) -- ({\Axisy*cos(90)},{\Axisy*sin(90)}) node [right] {$I_{\scriptscriptstyle  C}$};

       % сам вектор
        \draw[thick, red,->] (0,0) -- ({\V*cos(\alfa)},{\V*sin(\alfa)}) node (V) {} node[below right] {$\vec{i}$};
	\draw[thick, blue,->] (0,0) -- ({\UU*cos(\betaa)},{\UU*sin(\betaa)}) node (U) {} node[right] {$\vec{u}_\text{сети}$};

	\draw (U.center) --++ ({-\V*sin(\alfa)/2},{\V*cos(\alfa)/2}) node (Uinvert) {};
	\draw[thick, blue,->] (0,0) -- (Uinvert.center) {} node[right] {$\vec{u}_\text{инвертора}$};
        % перпендикулярные проекции
%        \draw[dashed] ({\UU*cos(\betaa)},{\UU*sin(\betaa)}) to[short,-] ({\UU*cos(\betaa)},0) node[below] {$u_{\scriptscriptstyle AB}$};
%        \filldraw[color=white, draw=black] ({\UU*cos(\betaa)},0)  circle (0.56mm);

       %проекция на сопряженную ось e^{AB}
%        \newcommand{\VAB}{\V*cos(\gammaa-90-\alfa)} % проекция на ось V^AB
%        \draw[dashed]  (V.center) -- ({\VAB/cos(\gammaa-90)}, 0) node[below] {$i^{\scriptscriptstyle AB}$};
%        \filldraw[color=white, draw=black] ({\VAB/cos(\gammaa-90)}, 0)  circle (0.56mm);

%        \draw[dashed]  (V.center) -- ({\VAB*cos(\gammaa-90)}, {\VAB*sin(\gammaa-90)}) node[above=1.3mm] {$i_{\scriptscriptstyle (-A)}$};
%        \filldraw[color=white, draw=black]  ({\VAB*cos(\gammaa-90)}, {\VAB*sin(\gammaa-90)})  circle (0.56mm);



%        \newcommand{\Ubc}{(\UU*cos(\betaa)*cos(\gammaa) + \UU*sin(\betaa)*sin(\gammaa))} % проекция на ось Vbc
%        \draw[dashed] (U.center) -- ({\Ubc*cos(\gammaa)},{\Ubc*sin(\gammaa)}) node[right=1.8mm] {$u^{\scriptscriptstyle BC}$};
%        \filldraw[color=white, draw=black]  ({\Ubc*cos(\gammaa)},{\Ubc*sin(\gammaa)}) circle (0.56mm);

        % проекции вдоль оси X_1
%        \draw[dashed]  (V.center) -- ({-\V*sin(\alfa)*tan(\gammaa-90)} , {\V*sin(\alfa)}) node[left] {$i^{\scriptscriptstyle BC}$};
%        \filldraw[color=white, draw=black]  ({-\V*sin(\alfa)*tan(\gammaa-90)} , {\V*sin(\alfa)}) circle (0.56mm);
%        \draw[dashed]  (V.center) -- (0, {\V*sin(\alfa)}) node[above right] {$i_{\scriptscriptstyle C}$};
%        \filldraw[color=white, draw=black]  (0, {\V*sin(\alfa)}) circle (0.56mm);

\end{circuitikz}
\end{frame}


\begin{frame}
\frametitle{\smallрезультат}
	В системе управления инвертора 
	\begin{itemize}
		\itemизлишен переход в декартову систему
		\itemиспользуем измеренные величины с минимумом преобразований 
	\end{itemize}
\end{frame}

\begin{frame}
\frametitle{\smallлитература}
	\vspace{-0.4cm}
	\small{
\begin{thebibliography}{7}
	\bibitem{Gorev}Горев А.А. Переходные процессы синхронной машины. -- М.,Л., Гос. энергетическое изд., 1950. -- 551 c.
        \bibitem{Sokolovsky}Соколовский Г.Г. Электроприводы переменного тока с частотным регулированием: Учебник для студ. высш.учеб.заведений.
                -- М. «Академия», 2007 - 272 с.
	\bibitem{Borisenko}Борисенко А.И., Тарапов И.Е. Векторный анализ и начала тензорного исчисления. -- М. «Высшая школа», 1966. -- 252 с.
        \bibitem{Prokshin}Илюшин А.Г., Маслов И.А., Прокшин А.Н. и др. О системах координат для математического описания систем управления электропривода. --
                Сборник докладов 71-й научно-технической конференции ППС, СПб, 2018, с.172
	
%	\bibitem{Proshivka}Заливка прошивки в STM32 через USB \url{https://habr.com/post/403007/}
%	\bibitem{Zagruzchik}Программа-загрузчик \url{github.com/rogerclarkmelbourne/Arduino\_STM32}
%	\bibitem{MexBios}Мехбиос \url{http://www.mechatronica-pro.com}
%	\bibitem{source} Исходный код блока \url{https://github.com/trot-t/RemoteLabs}
%	\bibitem{ControlSUITE} \url{https://www.ti.com/tool/CONTROLSUITE}
		%описание на русском https://habr.com/post/403007/\\
%на английском %http://www.rogerclark.net/stm32f103-and-maple-maple-mini-with-arduino-1-5-x-ide/
%
%       Сама  программа загрузчик -- \\
%{\small https://github.com/rogerclarkmelbourne/Arduino\_STM32}

\end{thebibliography}
}
\end{frame}
\end{document}

